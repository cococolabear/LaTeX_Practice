% \documentclass{article}
% \usepackage{xeCJK, fontspec, xunicode, xltxtra}  
% \setCJKmainfont{Hiragino Sans GB}  

% \title{Title}
% \author{}

% \begin{document}

% \maketitle{}

% \section{Introduction}

% This is where you will write your content. 在这里写上内容。

% \end{document


% \documentclass{article}  
% \title{My first Latex document} 
% \author{Yingshan Li} 
% \date{8/26/2018} 
% \begin{document} 
% \maketitle 

% Hello world! 

% \end{document}


\documentclass{article}  
\title{My first Latex document} 
\author{Yingshan Li} 
\date{8/26/2018} 
\begin{document} 
\maketitle 
\tableofcontents 


% \section{} 
% \subsection{} 
% \subsubsection{}

\paragraph{} 
dsfsdfds
\subparagraph{} 
sfdsfsd
% \subsubparagraph{}
fsfdsfs

Einstein 's $E=mc^2$.   %行内公式
\[ E=mc^2. \]   %行间公式 自动居中

%插入多个公式
\begin{displaymath} 
S_{n+1} = S_{n} + S_{n},  
S_{n}=1=2^{n} 
\end{displaymath}


%对行间公式进行编号
\begin{equation} 
E=mc^2 
\end{equation}
\begin{equation} 
E=mc^2 
\end{equation}
\begin{equation} 
E=mc^2 
\end{equation}
\begin{equation} 
E=mc^2 
\end{equation}




Hello world! \\
vgdff

\end{document}


